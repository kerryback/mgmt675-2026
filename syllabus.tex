\documentclass[11pt]{article}
\usepackage[margin=1in]{geometry}
\usepackage{graphicx}
\usepackage[colorlinks=true,linkcolor=blue,urlcolor=blue]{hyperref}
\setlength{\parindent}{0pt}
\begin{document}

\begin{minipage}[c][3cm][c]{0.5\textwidth}
\Large\textbf{MGMT 675\\ Generative AI for Finance}
\end{minipage}
\hfill
\begin{minipage}[c][3cm][c]{0.4\textwidth}
\includegraphics[width=\textwidth]{rice-business-transparent-final.png}
\end{minipage}

\vspace{0.5cm}
\hrule

\subsection*{Instructor}
Kerry Back\\
J. Howard Creekmore Professor of Finance and Professor of Economics\\
325 McNair Hall\\
\href{mailto:kerry.e.back@rice.edu}{kerry.e.back@rice.edu}

\subsection*{Learning Objectives}

The emergence of generative AI has been reshaping financial workflows and careers. This course prepares MBA students to leverage AI for financial analysis.  The course is organized around the folowing principles:

What you will learn:
\begin{enumerate}
\item How to work with AI to do financial analysis
\item How custom chatbots and AI agents work 
\item How to work with AI to build custom chatbots and AI agents for financial analysis
\end{enumerate}


Item \#1 is essential today. Item \#2 is valuable today. The purpose of \#3 is
\begin{enumerate}\renewcommand{\labelenumi}{3.\arabic{enumi}}
\item More practice on  \#1
\item A deeper understanding of  \#2
\item You might end up at a small firm and be the AI expert!
\end{enumerate}


\subsection*{AI Tool}

We are going to use Anthropic's Claude for the course.  Claude Opus and Sonnet are the best coding LLMs.  Also, Anthropic recently released Excel integration, so Claude can generate fully functioning, nicely formatted Excel workbooks. In addition, Anthropic created the Model Context Protocol (MCP) and Claude Skills, which substantially simplify the creation of AI agents.  In fact, Claude Desktop or Claude Code can be used as general purpose agents.  OpenAI also uses MCP and has a coding agent along the same lines as Claude Code, but Anthropic is a bit ahead in this area at the moment.

\subsection*{Evaluation}

Grades will be based on four elements (25\% each).  The first three are \textbf{group projects}.

\begin{enumerate}\setlength{\itemsep}{0pt}
\item AI cost of capital exercise: calculation, visualization, report generation 
\item  Cost of capital app 
\item Valuation Skill for Claude
\item Seated exam with AI allowed 
\end{enumerate}
For each of the group projects, the deliverables are:
\begin{itemize}\setlength{\itemsep}{0pt}
\item A chat/app/agent
\item A two-page Word doc explaining development and use
\end{itemize}

\subsubsection*{Honor Code}
The Rice University honor code applies to all work in this course. Use of generative AI is of course permitted.

\subsubsection*{Disability Accommodations}
Any student with a documented disability requiring accommodations in this course is encouraged to contact me outside of class. All discussions will remain confidential. Any adjustments or accommodations regarding assignments or the final exam must be made in advance. Students with disabilities should also contact Disability Support Services in the Allen Center.

\vspace{1cm}
\hrule
\vspace{0.5cm}

\begin{center}
\textbf{\Large  Schedule}
\end{center}

\subsubsection*{Week 1: Introduction}
\begin{itemize}\setlength{\itemsep}{0pt}
\item Collaborating with AI: Claude + Excel for valuation
\begin{itemize}
\item Case study: \href{https://hbsp.harvard.edu/product/W11058-PDF-ENG}{Valuing Walmart 2010}
\end{itemize}
\item Corporate implementations of AI
\begin{itemize}

\item Case study: \href{https://hbsp.harvard.edu/product/HEC382-PDF-ENG}{Implementing AI at Deloitte}
\item Other reading:
\begin{itemize} 
    \item \href{https://www.cnbc.com/2025/09/30/jpmorgan-chase-fully-ai-connected-megabank.html}{CNBC Report on JPMorgan Chase's LLM Suite}
\item \href{https://mlq.ai/media/quarterly_decks/v0.1_State_of_AI_in_Business_2025_Report.pdf}{State of AI in Business 2025}
\item
\href{https://papers.ssrn.com/sol3/papers.cfm?abstract_id=5188231}{Generative AI Reshaping Teamwork and Expertise}
\end{itemize}
\end{itemize}
\end{itemize}

\subsubsection*{Week 2: Vibe Coding for Data Analysis}
\begin{itemize}\setlength{\itemsep}{0pt}
\item Claude Desktop and Google Colab
\item Cleaning, sorting, filtering and aggregating
\item Data visualization
\item Generating Word docs and PowerPoint decks 
\item Installing Python locally and Claude Code
\end{itemize}

\subsubsection*{Week 3: Vibe Coding for Financial Analysis}
\begin{itemize}\setlength{\itemsep}{0pt}
\item Retirement planning simulation
\item Mean-variance portfolio optimization
\item Mutual fund performance evaluation 
\end{itemize}

\subsubsection*{Week 4: Custom Chatbots and Apps}
\begin{itemize}\setlength{\itemsep}{0pt}
\item Building and running Streamlit apps
\item Examples: retirement planning and asset allocation
\item Custom chatbots: API calls, system prompts, and RAG
\item Building custom chatbots as Streamlit apps
\end{itemize}

\subsubsection*{Week 5: Automating DCF Analysis}
\begin{itemize}\setlength{\itemsep}{0pt}
\item Converting rules into apps
\item Configuring apps as chatbot tools
\item Fine-tuning output (Excel, PowerPoint, Word) 
\item Collaborating with the chatbot on assumptions
\item Group project 2 due
\end{itemize}

\subsubsection*{Week 6: Deployment and Databases}
\begin{itemize}\setlength{\itemsep}{0pt}
\item Alternatives for deploying apps/chatbots/agents
\item Deploying databases
\item Creating database agents
\item Wrap-up
\item Group project 3 due
\end{itemize}



\end{document}